\documentclass{article}
\usepackage[utf8]{inputenc}
\usepackage{listings}
\usepackage{color}
\usepackage[left=2.5cm,top=2.5cm,right=2.5cm,bottom=2.5cm]{geometry} 


\title{Exercícios de optimização}
\author{juliane.marubayashi@gmail.com}
\date{December 2019}

\begin{document}

\definecolor{keywords}{RGB}{255,0,90}
\definecolor{comments}{RGB}{0,0,113}
\definecolor{red}{RGB}{160,0,0}
\definecolor{green}{RGB}{0,150,0}
 
\lstset{language=Python, 
        backgroundcolor=\color{white},   % choose the background color
        basicstyle=\footnotesize,        % size of fonts used for the code
        breaklines=true,                 % automatic line breaking only at whitespace
        captionpos=b,                    % sets the caption-position to bottom
        commentstyle=\color{green},    % comment style
        escapeinside={\%*}{*)},          % if you want to add LaTeX within your code
        keywordstyle=\color{blue},       % keyword style
        stringstyle=\color{red}     % string literal style
}
\maketitle

Os exercícios neste documento são baseados nos dados em sala de aula
\section{Exercicio: secção áurea}
Utilizando o método da secção áurea, ache o valor mínimo e máximo da equação a seguir no intervalo [-1,0]:
$$ f(x) = (2x+1)^2 -5cos(10x)$$ 

\section{Exercicio: gradiente}
Utilizando o método do gradiente encontre o mínimo da função a seguir no com h = 1, x0 = 1 e y0 = 1: 
$$ f(x,y) = y^2 - 2xy - 6y + 2x^2 +12$$

\section{Exercicio: Quádrica}
Utilizando o método da quádrica, encontre o mínimo da equação seguinte considerando x0= 0 e y0= 0:
$$ f(x,y) = sin(y)+ \frac{y^2}{4} + cos(x) + \frac{x^2}{4} -1$$
Resposta: x = 0, y = -1,02987, f(x,y) = -0,59207

\section{Exercicio: Levemberg Marquardt}
Utilizando o método de Levemberg Marquardt, encontre o mínimo da equação a seguir considerando lambda = 1, x0 = 1, y0 = 1: 
$$ f(x,y) = y^2-2xy -6y +2x^2+12$$
Resposta: x = 2,9692650, y = 5,950269, f(x,y) = -5,998


\newpage
\section{Resposta - exercicio 1 - python}

\lstinputlisting{Opt_1.py}

\newpage

\section{Resposta - exercicio 2 - python}

\lstinputlisting{Opt_2.py}

\newpage
\section{Resposta - exercicio 3 - python}

\lstinputlisting{Opt_3.py}

\newpage 
\section{Resposta - exercicio 4 - python}
\lstinputlisting{Opt_4.py}
\end{document}
