\documentclass{article}
\usepackage[utf8]{inputenc}
\usepackage{listings}
\usepackage{color}
\usepackage[left=2.5cm,top=2.5cm,right=2.5cm,bottom=2.5cm]{geometry} 

\definecolor{dkgreen}{rgb}{0,0.6,0}
\definecolor{gray}{rgb}{0.5,0.5,0.5}
\definecolor{mauve}{rgb}{0.58,0,0.82}

\lstset{frame=tb,
  language=Java,
  aboveskip=3mm,
  belowskip=3mm,
  showstringspaces=false,
  columns=flexible,
  basicstyle={\small\ttfamily},
  numbers=none,
  numberstyle=\tiny\color{gray},
  keywordstyle=\color{blue},
  commentstyle=\color{dkgreen},
  stringstyle=\color{mauve},
  breaklines=true,
  breakatwhitespace=true,
  tabsize=3
}


\title{Exercícios de optimização}
\author{juliane.marubayashi@gmail.com}
\date{December 2019}

\begin{document}

\maketitle

Os exercícios neste documento são baseados nos dados em sala de aula
\section{Exercicio: secção áurea}
Utilizando o método da secção áurea, ache o valor mínimo e máximo da equação a seguir no intervalo [-1,0]:
$$ f(x) = (2x+1)^2 -5cos(10x)$$ 

\section{Exercicio: gradiente}
Utilizando o método do gradiente encontre o mínimo da função a seguir no com h = 1, x0 = 1 e y0 = 1: 
$$ f(x,y) = y^2 - 2xy - 6y + 2x^2 +12$$

\section{Exercicio: Quádrica}
Utilizando o método da quádrica, encontre o mínimo da equação seguinte considerando x0= 0 e y0= 0:
$$ f(x,y) = sin(y)+ \frac{y^2}{4} + cos(x) + \frac{x^2}{4} -1$$
Resposta: x = 0, y = -1,02987, f(x,y) = -0,59207

\section{Exercicio: Levemberg Marquardt}
Utilizando o método de Levemberg Marquardt, encontre o mínimo da equação a seguir considerando lambda = 1, x0 = 1, y0 = 1: 
$$ f(x,y) = y^2-2xy -6y +2x^2+12$$
Resposta: x = 2,9692650, y = 5,950269, f(x,y) = -5,998


\newpage
\section{Resposta - exercicio 1 - python}

\begin{lstlisting}
    import math as m


    def f(x):
        return (2 * x + 1) ** 2 - 5 * m.cos(10 * x)
    
    
    def aurea_max(x1, x2):
        b = (m.sqrt(5) - 1) / 2
        a = b * b
        for i in range(30):
            x3 = x1 + a * (x2 - x1)
            x4 = x1 + b * (x2 - x1)
            if f(x3) > f(x4):
                x2 = x4
                x4 = x3
            else:
                x1 = x3
                x3 = x4
        return [x1, x2, x3, x4]
    
    
    def aurea_min(x1, x2):
        b = (m.sqrt(5) - 1) / 2
        a = b * b
        for i in range(30):
            x3 = x1 + a * (x2 - x1)
            x4 = x1 + b * (x2 - x1)
            if f(x3) < f(x4):
                x2 = x4
                x4 = x3
            else:
                x1 = x3
                x3 = x4
        return [x1, x2, x3, x4]
    
    
    print(aurea_max(-1, 0))
    #Expected result:
    #[-0.31113734222759837, -0.3111368047370985, -0.311137136924496, -0.311137136924496]
    print(aurea_min(-1, 0))
    #Expected result: 
    #[-0.6262978964093815, -0.6262973589188816, -0.6262976911062791, -0.6262976911062791]
\end{lstlisting}

\newpage

\section{Resposta - exercicio 2 - python}
\begin{lstlisting}
import math as m


def f(x, y):
    return y * y - 2 * x * y - 6 * y + 2 * x * x + 12


def dfx(x, y):
    return 4 * x - 2 * y


def dfy(x, y):
    return 2 * y - 2 * x - 6


def gradiente(xn, yn, h):
    for i in range(30):
        x = xn - h * dfx(xn, yn)
        y = yn - h * dfy(xn, yn)
        if f(x, y) < f(xn, yn):
            h *= 2
            xn = x
            yn = y
        else:
            h /= 2
    return [x, y]


print(gradiente(1, 1, 1))
#Expected result: 
#[2.9765625, 5.984375]
\end{lstlisting}
\newpage
\section{Resposta - exercicio 3 - python}
\begin{lstlisting}
    import math as m


def f(x, y):
    return m.sin(y) + y * y / 4 + m.cos(x) + x * x / 4 - 1


def dfx(x, y):
    return x / 2 - m.sin(x)


def dfy(x, y):
    return m.cos(y) + y / 2


def dfxx(x, y):
    return 1 / 2 - m.cos(x)


def dfyy(x, y):
    return 1 / 2 - m.sin(y)


def dfxy(x, y):
    return 0


def dfyx(x, y):
    return 0


def quadratica(xn, yn):
    for i in range(30):
        det = dfyy(xn, yn) * dfxx(xn, yn) - dfyx(xn, yn) * dfxy(xn, yn)
        x = xn - (dfyy(xn, yn) * dfx(xn, yn) - dfyx(xn, yn) * dfx(xn, yn)) / det
        y = yn - (-dfxy(xn, yn) * dfx(xn, yn) + dfxx(xn, yn) * dfy(xn, yn)) / det
        xn = x
        yn = y
    return [x, y]


print(quadratica(0, 0))

\end{lstlisting}

\newpage 
\section{Resposta - exercicio 4 - python}
\begin{lstlisting}
    def f(x, y):
    return y * y - 2 * x * y - 6 * y + 2 * x * x + 12


def dfx(x, y):
    return 4 * x - 2 * y


def dfy(x, y):
    return 2 * y - 2 * x - 6


def dfxx(x, y):
    return 4


def dfyy(x, y):
    return 2


def dfyx(x, y):
    return -2


def dfxy(x, y):
    return -2


# xn+1  xn - invert(hessiana).gradient + lambda.gradiente
def levenberg(x, y, lamb):
    x_ant = x
    y_ant = y
    for i in range(20):
        det = dfxx(x_ant, y_ant) * dfyy(x_ant, y_ant) - dfxy(x_ant, y_ant) * dfyx(x_ant, y_ant)
        x = x_ant - (dfyy(x_ant, y_ant) * dfx(x_ant, y_ant) - dfxy(x_ant, y_ant) * dfy(x_ant, y_ant)) / det - lamb * (
            dfx(x_ant, y_ant))
        y = y_ant - (-dfxy(x_ant, y_ant) * dfx(x_ant, y_ant) + dfxx(x_ant, y_ant) * dfy(x_ant, y_ant)) / det - lamb * (
            dfy(x_ant, y_ant))
        if (x - x_ant <= 0) and (y - y_ant <= 0):
            x_ant = x
            y_ant = y
        lamb /= 2
        print(x, y)
    return [x, y]


r = levenberg(1, 1, 1)
print(f(r[0], r[1]))

\end{lstlisting}
\end{document}
